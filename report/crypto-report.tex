\documentclass[pdflatex, a4paper,12pt]{article}
%\documentclass[a4paper,10pt]{scrartcl}

%some packages, mostly from malucrawl report
\usepackage[pdftex]{graphicx}
\usepackage[utf8x]{inputenc}
\usepackage{cite}
\usepackage{hyperref}
\usepackage{listings}
\usepackage{nomencl}
\usepackage{verbatim}
\usepackage{tabularx}

\providecommand{\e}[1]{\ensuremath{\times 10^{#1}}}

\begin{document}
\begin{center}
{\LARGE ELEC6032: Cryptography Coursework}\\[1em]

Author: Chris Orchard\\
\end{center}

TODO: an amount of intro?

\section{Q1}

Before attempting to decrypt the cipher, it was first necessary to transcribe
the hexadecimal string that formed the ciphertext into a format that was machine
readable. This was done by hand by copying the string into a binary hexadecimal
editor, and then comparing the two texts to ensure accuracy. In the process of
transcribing the ciphertext, it became obvious that there was a pattern in the
ciphertext aligned to to the byte boundaries and more likely aligned to a
multiple of four bytes, as there is a clearly visible pattern of similar bytes
in the columns of the cipher text, with the with in the provided image 28
characters. Upon performing autocorrelation of the ciphertext it was evident
that a likely key length was 32 bits, which correlates with the previous
observations. This however is one bit longer than the period of a degree 5 LFSR,
which the hint describes as the basis for the outer layer of encryption.

To break the LFSR cipher, a cracker was built using a C++ cryptographic library
\footnote[1]{https://github.com/mvy/Cryptographic-C---Toolkit}, that iterated
through all possible states and characteristic functions of an LFSR and tried to
match the output to the given first two characters \verb+Ur+. This program
(included with this report as lfsr.cpp) successfully found an LFSR that provided
the appropriate plaintext, and the text derived is as follows:

\begin{quote}
\verbatiminput{fourth.tex}
\end{quote}

This does not look like the output of a possible alphanumeric cipher and in
fact it appears that every fourth character is incorrect, or at least is not an
alphanumeric character. It appears that an incorrect assumption has been made
about the key, that the key consists of 31 bits of LFSR output followed by a
single bit. Given the structure of ASCII it cannot be the case that the last bit
is incorrect, otherwise the output would look convincingly like alphanumeric text. 

Continuing to run with the assumption that all 31 bits of the LFSR are being
used, the next step is to find out where the last bit is placed. As the
implementation in \verb+lfsr.cpp+ computes the LFSR keystream on a byte by byte
basis, the first obvious place to try would be at the other end of the byte
meaning that the keystream would consist of 24 bits of LFSR, one arbitrary bit
(that is set to 0 in this case), followed by the last 7 bits of LFSR stream. The
output of this configuration is as follows:

\begin{quote}
\verbatiminput{decrypt.tex}
\end{quote}

The output of this stage now looks much more like the output of an alphanumeric
cipher. The next stage in decrypting the ciphertext is to apply frequency
analysis. Table (cite) shows the top 8 frequencies of letters in the English
language beside the top 8 frequencies in the ciphertext, derived from tools at
(cite).
\begin{center}
%\begin{tabularx}{0.8\textwidth}{cc|cc}
\begin{tabular}{cc|cc}
    \multicolumn{2}{c|}{English} & \multicolumn{2}{c}{Ciphertext} \\
    \hline
    Letter & Frequency(\%) & Letter & Frequency(\%) \\
    \hline
    E                & 12.7         & R      & 14.9  \\
    T                & 9.1          & G      & 8.6   \\
    A                & 8.2          & B      & 8.1   \\
    O                & 7.5          & F      & 7.0   \\
    I                & 7.0          & N      & 7.0   \\
    N                & 6.7          & V      & 6.6   \\
    S                & 6.3          & U      & 6.6   \\
    H                & 6.1          & A      & 6.4   \\
\end{tabular}
\end{center}

The frequencies indicate that the second stage cipher is probably a
monoalphabetic substitution cipher. To decrypt the cipher a second tool was
written in C++ \verb+monosubs.ccp+ that takes the substitution alphabet as
arguments and then displays the ciphertext and the plaintext. After replacing
the top 8 frequency characters in table (cite), it was then possible to do the
rest of the decryption by hand, resulting in the following output:

%\begin{minipage}{3in}
\begin{quote}
    \verbatiminput{cipherplain.tex}
\end{quote}
%\end{minipage}

The final substitution alphabet is as follows, shown as the arguments to the
substitution helper:

%\begin{quote}
\begin{verbatim}
./alpha n o p d r s t u v w k y m a b c d e f g h i w k l m
\end{verbatim}
%\end{quote}
By rearranging the unused characters to a more familiar order, it becomes
obvious that the second stage encryption is in fact ROT-13.

\section{Q2}
\section{Q3}

This cipher takes form of a string of alphabetic characters with capitalisation
and some punctuation included. To start the process of decrypting the text, frequency
analysis was performed using the tool hosted at(cite). The top 8 results of
analysis are shown below.

\begin{center}
%\begin{tabularx}{0.8\textwidth}{cc|cc}
\begin{tabular}{cc|cc}
    \multicolumn{2}{c|}{English} & \multicolumn{2}{c}{Ciphertext} \\
    \hline
    Letter & Frequency(\%) & Letter & Frequency(\%) \\
    \hline
    E                & 12.7         & V      & 13.8  \\
    T                & 9.1          & R      & 9.2   \\
    A                & 8.2          & G      & 8.0   \\
    O                & 7.5          & L      & 7.6   \\
    I                & 7.0          & M      & 7.5   \\
    N                & 6.7          & H      & 6.3   \\
    S                & 6.3          & Z      & 6.3   \\
    H                & 6.1          & I      & 5.9   \\
\end{tabular}
\end{center}

The frequencies appear to match what would be expected for a simple
monoalphabetic substitution cipher. As in question one, the substitution helper
was used to solve the cipher.
Firstly the top 8 frequencies were swapped, and then the rest of the cipher was
completed by manual analysis. However, the letter frequencies for this cipher
are not those of English, as the cipher text contains multiple single capital
Rs which the frequency table shows should be substituted to T, but in English
can only correspond to the letter I. Eventually the plaintext was derived
primarily by solving the 2-3 letter words until the correct substitution for T
was found. The plaintext is as follows:

\begin{quote}
    Dear Sarah I am writing to apply for the programmer position advertised in the Times Union. as requested I am enclosing a completed job application, my certification my resume and three references The opportunity presented in this listing is very interesting and I believe that my strong technical experience and education will make me a very competitive candidate for this position The key strengths that I possess for success in this position include I have successfully designed, developed, and supported live use applications I strive for continued excellence I provide exceptional contributions to customer service for all customers With a BS degree in Computer Programming, I have a full understanding of the full life cycle of a software development project I also have experience in learning and excelling at new technologies as needed Please see my resume for additional information on my experience Thank you for your time and consideration I look forward to speaking with you about this employment opportunity You have now reached the end of this message 

\end{quote}    

The cipher for the text is as follows, and can be recognised as the Atbash
cipher as implemented for the Latin alphabet:

\begin{verbatim}
./alpha z y x w v u t s r q p o n m l k j i h g f e d c b a
\end{verbatim}

\section{Q4}


\end{document}
